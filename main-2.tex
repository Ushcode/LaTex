\documentclass{article}
\usepackage[utf8]{inputenc}
\usepackage[A4, portrait, margin=1in]{geometry}
\usepackage{amsfonts}
\usepackage{amsmath}
\DeclareMathOperator\cis{cis}
\DeclareMathOperator\Arg{Arg}
\setlength{\parindent}{1em}
\setlength{\parskip}{1em}
\usepackage[dvipsnames]{xcolor}
\definecolor{myred}{rgb}{1, 1, 1}
\usepackage{empheq}

\newlength\mytemplen
\newsavebox\mytempbox

\makeatletter
\newcommand\mybluebox{%
    \@ifnextchar[%]
       {\@mybluebox}%
       {\@mybluebox[0pt]}}

\def\@mybluebox[#1]{%
    \@ifnextchar[%]
       {\@@mybluebox[#1]}%
       {\@@mybluebox[#1][0pt]}}

\def\@@mybluebox[#1][#2]#3{
    \sbox\mytempbox{#3}%
    \mytemplen\ht\mytempbox
    \advance\mytemplen #1\relax
    \ht\mytempbox\mytemplen
    \mytemplen\dp\mytempbox
    \advance\mytemplen #2\relax
    \dp\mytempbox\mytemplen
    \colorbox{myred}{\hspace{1em}\usebox{\mytempbox}\hspace{1em}}}


\title{Complex Analysis}
\author{Oisín Peppard}

\begin{document}

\maketitle

\textcolor{Gray}{\subsection*{Lecture 1 - 8 Sep}}

\section{Introduction and review}

\begin{equation}
    \log{z} = \ln{|z|} + i \arg{z}
\end{equation}
\begin{equation}
    \arg{z} = \Arg{z} + 2\pi i \mathbb{Z}
\end{equation}

\begin{flushleft}
 multiplication operator is a \textbf{bifunctor}, a map between elements provides a map between sets. 

\par A \textbf{Branch} of a multi valued function is any $f$ s.t. 


\begin{itemize}
    \item $f$ is continuous
    \item $f(z) \in F(z) $
    \item $f(z) : z \in $ some region $D$
\end{itemize}

Example, a branch of $f(z)$ over $D = \mathbb{C} \setminus \mathbb{R}_{<0}$ This includes a cut of the negative real numbers. To account for discontinuity at $\pm{\pi}$

\par
Holomorphic functions are those that infinitely differentiable on a given domain.
\par
An I Isometric map preserves distance and angle, while a conformal map preserves angle alone.
\end{flushleft}

\textcolor{gray}{\subsection*{Lecture 2 - 12 Sep}}
\section{Conformal Maps}
\begin{flushleft}
To discuss angles between intersecting arcs, we first define an \textbf{arc} as a path, or a trajectory parameterised by time. These may be closed, self intersecting and even have corners or cusps. They cannot however be discontinuous.

\begin{equation}
\textrm{Arc } \gamma:[a,b] \to \mathbb{C}
\end{equation}

An arc is closed if $\gamma (a) = \gamma (b) $
\par
Reparameterisation is via a homeomorphism $phi$ between $[a,b] \to [\alpha, \beta]$. In this case we have:

\begin{equation}
    \phi \circ \gamma : [\alpha,\beta] \to \mathbb{C}
\end{equation}
A \textbf{curve} is an equivalence class of arcs modulo reparameterisation. \par An arc is \textbf{regular} if it is of class 1 i.e. if it is differentiable. \par Arcs are equivalent at a point $a$ if $\gamma_{1} (a) = \gamma_{2}(a)$ and $\gamma_1 ' (a) = \gamma_2 '(a)$

A tangent vector at $z_0$ is an equivalence class of arcs $\gamma(a) = z_0$ s.t. :
\begin{equation}
\gamma_1 \sim \gamma_2 \iff |\gamma_1 (t) - \gamma_2 (t)| = o(t) ; t\to 0 
\end{equation}

Here, $|\gamma_1 (t) - \gamma_2 (t)|$ is a metric, and can be looked at for any submanifold or non - Euclidean space. $f(t) = o(t);t\to 0$ means $\frac{|f(t)|}{t} \to 0$. Similar to $f(t) = O(g(t));t\to 0 $ if $ \frac{|f(t)|}{g(t})$ is bounded.
\par\begin{equation}
    f: \Omega \to \Omega' \textrm{ open in $\mathbb{C}$}
\end{equation}
$f$ is R-Differentiable at $z_0$ if $\exists$ a linear transformation $L :\mathbb{R}^n \to \mathbb{R}^m$ s.t.
\begin{equation}
    f(z) = f(z_0) + L(z-z_0) + O_{z\to z_0}(|z-z_0|)
\end{equation}
\par \textbf{Lemma} if $L$ is linear and $O(|z-Z_0|)$, then $L = 0$
\par if $f$ is R-Differentiable, $L = df_{z_0} : \mathbb{R}^n \to \mathbb{R}^m $ is the differential of f at $z_0$

\begin{equation}
    df = \sum^n{f_{x_j} dx_j} \iff df(\xi) = \sum^n{f_{x_j} dx_j}(\xi) = \sum^n {f_{x_j}}\xi_j \textrm{ with } \xi = (\xi_1,...,\xi_n)
\end{equation}

\begin{empheq}[box={\mybluebox[5pt]}]{equation}
\textrm{In} \mathbb{C}, df = f_z dz + f_{\bar{z}} d\bar{z} \textrm{and} dz(\xi) = \xi, d_{\bar{z}}(\xi) = \bar{\xi} 
\end{empheq}

\par if f is $\mathbb{C}$-Differentiable at  $z_0$, it is R-Differentiable and $f_{z_0} = 0$. This happens $\iff f$ is R-diff and $df$ is $\mathbb{C}$-Linear.

\section{Linear complex structure}
\par for an arbitrary vector space $V$ over a field $\mathbb{R}$, we may have a complex structure that is $\mathbb{R}$-linear. This is given by a linear transformation $J:V \to V$ where $J^2 = -\mathbb{1}$. Complex Structure is an automorphism of the vector space, equipping a real $V$ with this structure allows us creation of a complex vector space, that is a vector space over a field $\mathbb{C}$.
\par \textbf{Note} that every complex vector space may be equipped with a complex structure, but there is no such canonical structure. The fundamental \textit{example} of a complex structure is $\mathbb{C}^n \textrm{on the space} \mathbb{R}^{2n}$. This idea is useful in complex geometry and the definition of almost-complex and complex manifolds.

\par Some properties of this structure (to be proven as an exercise):
\begin{itemize}
    \item If $V$ is a $\mathbb{C}$ Vector space, then $J(v) = {iv: v \in V}$ is a complex structure. (Alternative realisation of $J$
    \item If $J$ is a complex structure on $V$ then $V$ becomes a $\mathbb{c}$ vector space with scalar multiplication defined as $(x + iy)v = xv +yJ(v)$
\end{itemize}



\end{flushleft}

\end{document}
